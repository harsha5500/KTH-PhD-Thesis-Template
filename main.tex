\documentclass[12pt]{book} % As pert template set the default font to 12pt.
\usepackage[a4paper,top=3.6cm,bottom=3cm,left=4.3cm,right=2.2cm,includefoot]{geometry} % based on the Word Template for KTH Thesis 2024.
\usepackage{graphicx} % Required for inserting images

%-----------------------------------------------------------------------
% DEBUGGING REMOVE FOR REAL PROJECTS
% Load blindtext package for generating LOREM IPSUM
%-----------------------------------------------------------------------
\usepackage{blindtext} 
%-----------------------------------------------------------------------

%-----------------------------------------------------------------------
% SETTING FONTS
%-----------------------------------------------------------------------
\usepackage{fontspec} % Required to specify custom fonts.
\usepackage[T1]{fontenc} % Enable T1 font encoding.

%-----------------------------------------------------------------------
% ***** OPEN FONT SUBSTITUTES. COMMENT HERE AND UNCOMMENT NEXT SECTION FOR LICENSED FONT
%-----------------------------------------------------------------------
\usepackage{libertine}
\usepackage{libertinust1math}
\usepackage[sfdefault]{arimo}

%-----------------------------------------------------------------------
% ***** UNCOMMENT THE FOLLOWING FOR USING LICENSED FONTS ******
% ***** ASSUMPTION GeorgiaPro folder and Figtree-Medium folder are in current directory.
% ***** NOTE: All font files have to be named 'GeorgiaPro'-'variant'-'.ttf' 
% ***** NOTE: ONLY TTF FONTS.  
%-----------------------------------------------------------------------

% Most normal text is GeorgiaPro
% \setmainfont{GeorgiaPro}[
%     Path= ./GeorgiaProFonts/,
%     Extension = .ttf,
%     UprightFont=*-Regular,
%     BoldFont=*-Bold,
%     ItalicFont=*-Italic,
%     BoldItalicFont=*-BoldItalic
% ]

% Chapter Headings are in Figtree Medium
% \setsansfont{Figtree-Medium}[
%     Path=./FigtreeFonts/,
%     Extension = .ttf,
%     UprightFont=Figtree-Medium,
%     BoldFont=Figtree-Bold,
%     ItalicFont=Figtree-Medium-Italic,
%     BoldItalicFont=Figtree-BoldItalic
% ]
%-----------------------------------------------------------------------

% In case a semi-bold font is explicity required
% \newfontfamily\semibf[
%     Path=./FigtreeFonts/,
%     Extension = .ttf,
%     UprightFont=Figtree-Semibold,
%     BoldFont=Figtree-Semibold-Italic,
%     ItalicFont=Figtree-Semibold-Italic,
%     BoldItalicFont=Figtree-Semibold-Italic
% ]{Figtree-Semibold}

%-----------------------------------------------------------------------
% HEADING FORMATTING
%-----------------------------------------------------------------------
\usepackage{titlesec} % Required to format headings.
% Guide for understanding the commands
% \titleformat{\section}  % which section command to format
%   {\fontsize{14}{16}\bfseries} % format for whole line
%   {\thesection} % how to show number
%   {1em} % space between number and text
%   {} % formatting for just the text
%   [] % formatting for after the text

% \titleformat{\subsection}  % which section command to format
%   {\fontsize{10}{12}\sffamily} % format for whole line
%   {\S\thesubsection} % how to show number
%   {2em} % space between number and text
%   {\color{blue}} % formatting for just the text
%   [] % formatting for after the text

% add \semibf to the formatting if semi-bold enabled to change the font explicitly.
% Format the Chapter Name
\titleformat {\chapter}{\fontsize{24pt}{24pt}\itshape\sffamily}{\thechapter} {0.5em}{}[\vspace{-1.25 em}]

% Format the Section Name
\titleformat {\section}{\fontsize{13pt}{13pt}\bfseries\sffamily}{\thesection} {0.5em}{}[]

% Format the Sub section Name
\titleformat {\subsection}{\fontsize{12pt}{12pt}\bfseries\sffamily}{\thesubsection} {0.5em}{}[]

% Format the SubSub section Name
\titleformat {\subsubsection}{\fontsize{12pt}{12pt}\itshape\bfseries\sffamily}{\thesubsubsection} {0.5em}{}[]

%-----------------------------------------------------------------------
% SECTION AND SUBSECTIONS
% Set the depth of sub sections to show in Contents to 3, to only show upto to sub sub section headings.
%-----------------------------------------------------------------------
\setcounter{secnumdepth}{3} % Add numbers to sub sections
%-----------------------------------------------------------------------

%-----------------------------------------------------------------------
% HEADERS AND FOOTERS
%-----------------------------------------------------------------------
\usepackage{fancyhdr}
%-----------------------------------------------------------------------

%-----------------------------------------------------------------------
% ENABLE ROMAN ENUMERATION
%-----------------------------------------------------------------------
\usepackage{enumitem}
%-----------------------------------------------------------------------

%-----------------------------------------------------------------------
% Abbreviations
%-----------------------------------------------------------------------
\usepackage[acronym,toc,nopostdot,nonumberlist,numberedsection]{glossaries}
%\usepackage{glossaries-extra}
\renewcommand*{\glsnamefont}[1]{\textmd{#1}\fontsize{13}{13}}
\renewcommand{\glossarypreamble}{\begin{sffamily}}
\renewcommand{\glossarypostamble}{\end{sffamily}}
\makeglossaries
\input{0.5-Abbreviations}
%-----------------------------------------------------------------------

%-----------------------------------------------------------------------
% Tables
%-----------------------------------------------------------------------
\usepackage{tabularx}
%-----------------------------------------------------------------------

%-----------------------------------------------------------------------
% TABLE OF CONTENTS
%-----------------------------------------------------------------------
\usepackage[titles]{tocloft}
% Adjust sectional unit title fonts in ToC
\renewcommand{\cftchapfont}{\sffamily\large}
\renewcommand{\cftsecfont}{\sffamily\normalsize}
\renewcommand{\cftsubsecfont}{\sffamily\normalsize}
%-----------------------------------------------------------------------

%-----------------------------------------------------------------------
% Paragraph spacing control
% Required for setting space between paragraphs.
%-----------------------------------------------------------------------
\usepackage[skip=10pt plus1.25pt]{parskip} 
%-----------------------------------------------------------------------

%-----------------------------------------------------------------------
% Bibliography
%-----------------------------------------------------------------------
\usepackage[backend=biber,style=ieee,sorting=ynt]{biblatex}
\addbibresource{8-References.bib}
% \usepackage[fixlanguage]{babelbib}
% \selectbiblanguage{german}
% \bibliographystyle % For babelbib to take effect, a bibliography style supported by it - one of babplain, babplai3, babalpha, babunsrt, bababbrv, and bababbr3 - must be used:

%-----------------------------------------------------------------------

\title{KTH PhD Thesis Template}
\author{Trimone}
\date{October 2024}

\begin{document}
% \maketitle

% ROMAN NUMERALS PART FOR PAGE NUMBERS

\pagestyle{fancy}
\fancypagestyle{plain}{
  \renewcommand{\headrulewidth}{0pt} % remove lines as well
  \renewcommand{\footrulewidth}{0pt}
}
\pagenumbering{roman}
\renewcommand{\headrulewidth}{0pt} % remove lines as well
\fancyhf{} % Clear all headers and footers
\fancyhf[FR]{\fontsize{9}{9}\sffamily\thepage} % Start page number 1
\renewcommand{\thechapter}{\Roman{chapter}} % set chapter numbers to Roman

\textit{Georgia Pro Italics 12pt.\\
Dedication page - Optional - If you don’t need a dedication page, just select everything from the top of this page to the next page, including the section break. Then press Delete.}

\chapter{Abstract}
\blindtext
\vspace{5em}
\\
\textbf{Keywords}\\
word 1, word 2, word 3

\include{0.2-Sammanfattning}
\include{0.3-Acknowledgements}
\chapter{List of publications}
\blindtext

\begin{enumerate}[label=\roman*.]
    \item Publication 1
    \item Publication 2
    \item Publication 3
    \item Publication 4
\end{enumerate}


% ACRONYMS
\printglossary[title=List of abbreviations, type=\acronymtype]

% CONTENTS
\addtocontents{toc}{\protect\pagestyle{empty}}
\addtocontents{toc}{\protect\thispagestyle{empty}}
\tableofcontents
\thispagestyle{empty}

% REST PAGE NUMBERINGS
\pagestyle{fancy}
\fancypagestyle{plain}{ %
  \renewcommand{\headrulewidth}{0pt} % remove lines as well
  \renewcommand{\footrulewidth}{0pt}
}
\pagenumbering{arabic}
\setcounter{page}{0}
\fancyhf{} % Clear all headers
\fancyhf[FR]{\fontsize{9}{9}\sffamily\thepage} % Start page number 1
\renewcommand{\thechapter}{\arabic{chapter}} % Set page numbers back to arabic
\setcounter{chapter}{0} %reset chapter counter

% MAIN THESIS CONTENT STARTS HERE WITH 
\chapter{Introduction}
\blindtext
\par
\blindtext

\section{My Section}
\blindtext

\subsection{My Sub Section}
\blindtext

\subsubsection{My Sub Sub Section}
\blindtext
\\
Given a set of numbers, there are elementary methods to compute its \acrlong{gcd}, which is abbreviated \acrshort{gcd}. This process is similar to that used for the \acrfull{lcm}.

\chapter{Materials and methods}
Text body Georgia Pro Regular 12pt. Paragraph spacing 1,25pt.
\section{Heading 2}
Acil dolor sed tatuercincip et, quis acipsummy nibh etuercilis auguercillut lum ipit la feugue ver suscil dolobor perilisl iurer se velisl ute modionsequam iniatue consecte velisl ea corercipit lam acipsum inim magna alis atie ver at.
\chapter{Background}

\begin{figure}[h]
    \centering
    \includegraphics[width=0.5\linewidth]{images/image.jpeg}
    \caption{Caption}
    \label{fig:enter-label}
\end{figure}

\include{4-Results}
\include{5-Conclusions}
\chapter{Points of Perspective}

Using \texttt{biblatex} you can display a bibliography divided into sections,  depending on citation type. Let's cite! Einstein's journal paper \cite{einstein} and Dirac's book \cite{dirac} are physics-related items. Next, \textit{The \LaTeX\ Companion} book, Donald Knuth's website \cite{knuthwebsite}, \textit{The Comprehensive Tex Archive Network} (CTAN) are \LaTeX-related items; but the others, Donald Knuth's items, \cite{knuth-fa} are dedicated to programming.

\chapter{References}
\printbibliography[heading=none,title={References}]

\end{document}

% LYCA TILL! 